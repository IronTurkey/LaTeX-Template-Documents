% -------------------------------------------------------------
%   ------ See bottom of document to edit parameters -------  |
% -------------------------------------------------------------




% ---------------------------------------------------------------------------------------------------
% || ------------------------------------ LOTS OF PACKAGES --------------------------------------- ||
% ---------------------------------------------------------------------------------------------------

% OK just a warning - PACKAGE ORDER MATTERS - for instance the maths packages
% not being at the top breaks the compilation. LaTeX shenanigans can be annoying.


% Maths Formatting Packages - they just add extra commands
\usepackage{amsmath}
\usepackage{array} % extends array formatting
\usepackage{amssymb}
\usepackage{nccmath}
\usepackage{esint} % more integrals
\usepackage{bm} % bold math text that's not awful
\usepackage{xfrac} % slanted frac \sfrac
\usepackage{bigints}
\usepackage{tabularx}
\usepackage{cancel}
\usepackage{mathtools}%idk whether this is needed but ¯\_(ツ)_/¯



% Document/Text Formatting Packages
\usepackage[a4paper, portrait, right=20mm, left=20mm, top=25mm, bottom=20mm, showframe=false]{geometry} % margins
\usepackage{booktabs} % tables - needed so that tables from latex table makers work
\usepackage{babel} % language options
\usepackage[utf8]{inputenc} % adds extra unicode characters
    \DeclareUnicodeCharacter{2212}{-}
\usepackage{dtk-logos}
\usepackage{multicol}
\usepackage{paracol}
\usepackage{fix-cm}
\usepackage{import}
\usepackage{csquotes} % it told me to
\usepackage{xcolor} % colours yay
\usepackage{xpatch}
\usepackage{accents} % text accents
\usepackage[none]{hyphenat} % hyphenation
\usepackage{tocloft} % table of contents control
    \renewcommand{\cftsecleader}{\cftdotfill{\cftdotsep}} % adds dots between section titles and page numbers in ToC
\usepackage[justification=centering, margin=10mm]{caption} 
\usepackage{subcaption} % improves captions for subfigures


\usepackage{placeins} % Adds the \FloatBarrier command
\usepackage[hidelinks, plainpages=false, pdfpagelabels]{hyperref} % Hyperlinks - yay we live in the modern world
\usepackage{url} % adds url linking
\usepackage{nameref} % adds ability to reference sections by name
\usepackage{bookmark}

\usepackage{calc} % adds \setlength and other commands
\usepackage{lastpage} % page number referencing

\usepackage[bottom]{footmisc} %footnote customizability, [bottom] forces footnotes to the bottom

\usepackage{fancyhdr} % fancy headers and footers

\usepackage[T1]{fontenc} 
\usepackage{lmodern} % fixes some font issues that happen sometimes
\usepackage{titling} % title stuff
\usepackage{titlesec} % title format customisation

\usepackage{changepage} % multicolumn stuff !!

\usepackage{abstract} % Allows abstract customization
    \addto{\captionsenglish}{\renewcommand{\abstractname}{}}   % clear the title
    \renewcommand{\absnamepos}{empty} % originally center -- don't want an abstract title or it's whitespace


% Physics Packages
\usepackage[arrowdel]{physics} % - \grad{} \div{} \curl{} \laplacian{}
\usepackage{siunitx} % adds si units \si{\unit} (it's hard to use at first, look up the documentation)


% Code Formatting Packages
\usepackage{listings} % code text environment


% Graphics Formatting Packages
\usepackage{tikz}
\usepackage{tikzscale}
    \usetikzlibrary{shapes,arrows,positioning}
    \usetikzlibrary{matrix,calc}
\usepackage{tikz-3dplot}
    \usetikzlibrary{calc,patterns,angles,quotes}
\usepackage{graphicx}
\usepackage{graphics}

\usepackage{pgfplots}
    % Dunno, it complains if this isn't here
    \pgfplotsset{compat=1.17}
\usepackage{pgf} % allows inclusion of pgf vector graphic images
\usepackage{adjustbox}




% ---------------------------------------------------------------------------------------------------
% || --------------------------------------- REFERENCING ----------------------------------------- ||
% ---------------------------------------------------------------------------------------------------

\usepackage[backend=biber, style=phys]{biblatex}
\addbibresource{references.bib}% Syntax for version >= 1.2
% the references.bib should be in the same directory as the tex file

% Harvard style=authoryear-ibid
% APA           style=apa
% MLA           style-mla
% IEEE          style=ieee
% Physics       style=phys
% Misc. Science style=science
%
% see https://www.overleaf.com/learn/latex/Biblatex_bibliography_styles



% ---------------------------------------------------------------------------------------------------
% || --------------------------------------- TEXT SPACING ---------------------------------------- ||
% ---------------------------------------------------------------------------------------------------

% overridden in customization at bottom
\setlength{\parindent}{0em}
\setlength{\parskip}{1em}


\newlength{\tablegap}
\setlength{\tablegap}{0.3em} % put in square brackets after i.e. \\[\tablegap]



% ---------------------------------------------------------------------------------------------------
% || ---------------------------------- CUSTOM MATHS COMMANDS ------------------------------------ ||
% ---------------------------------------------------------------------------------------------------


\setlength{\jot}{0.5em} % length between newlines of align environment

% Symbols and Notation
\newcommand{\ud}{\ \mathrm{d}}
\renewcommand{\d}{\mathrm{d}}
\newcommand{\Reals}{\mathbb{R}}
\newcommand{\Complexs}{\mathbb{C}}
\newcommand{\Integers}{\mathbb{Z}}
\newcommand{\Naturals}{\mathbb{N}}
\newcommand{\Rationals}{\mathbb{Q}}
\newcommand{\Primes}{\mathbb{P}}

\newcommand{\emf}{\mathcal{E}}

\newcommand{\deq}{\coloneqq} % defined as equal

% Size overriding - big things
\newcommand{\bfrac}[2]{\frac{\displaystyle #1}{\displaystyle #2}}
\newcommand{\bint}{\displaystyle \int}

% Vector Formatting
\newcommand{\hvec}[1]{\accentset{\rightharpoonup{}\vspace{-0.5mm}}{#1}} % harpoon above
\newcommand{\svec}[1]{\underaccent{\tilde}{#1}} % squiggle below
\newcommand{\bvec}[1]{\boldsymbol{\mathbf{#1}}} % math bold font

% -Long harpoon vector
\makeatletter
\newcommand*{\rightharpoonupfill@}{\arrowfill@\relbar\relbar\rightharpoonup}
\newcommand{\lhvec}{\mathpalette{\overarrow@\rightharpoonupfill@}}
\newcommand{\lvect}[1]{\bvec{\lhvec{#1}}}%final edit here
% -end Long harpoon vector

\newcommand{\vect}[1]{\bvec{\hvec{#1}}} % used vector format
\newcommand{\vhat}[1]{\bvec{\hat{#1}}} % used hat format

\newcommand{\vmod}[1]{\left\| #1 \right\|} % modulus notation

% Vector Calculus Operators
%\renewcommand{\grad}{\vect{\nabla}}
%\renewcommand{\div}{\vect{\nabla} \cdot}
%\renewcommand{\curl}{\vect{\nabla} \times}
%\renewcommand{\laplacian}{\bvec{\nabla}^2}
%\newcommand{\del}{\vect{\nabla}}

% projection operator
\DeclareMathOperator{\vectProj}{proj}
\newcommand{\proj}[2]{\vectProj_{#2} {#1}}

% Custom Matrix - better spacing
\newcommand{\Matrix}[1]{\renewcommand{\arraystretch}{1.25}
    \left[\hspace{0.25em}\begin{matrix}
    #1
\end{matrix}\hspace{0.25em}\right]
\renewcommand{\arraystretch}{1}}




% ---------------------------------------------------------------------------------------------------
% || ------------------------------- INTERNAL REFERENCING COMMANDS ------------------------------- ||
% ---------------------------------------------------------------------------------------------------

\newcommand{\secrefNumName}[1]{\mbox{\hyperref[#1]{\S}\ref{#1}\hyperref[#1]{:\hspace{1pt}}\nameref{#1}}} % ref with Num:Name
\newcommand{\secrefName}[1]{\mbox{\hyperref[#1]{\S\hspace{1pt}}\nameref{#1}}} % ref with just name
\newcommand{\secrefNum}[1]{\mbox{\hyperref[#1]{\S}\ref{#1}}} % ref with just num

\newcommand{\secref}[1]{\secrefNumName{#1}} % choose method of section reference


\newcommand{\eqnum}[1]{\ref*{#1}}


% Lower case commands
\newcommand{\appref}[1]{\mbox{\hyperref[#1]{appendix\hspace{2pt}}\ref{#1}}}
\renewcommand{\eqref}[1]{\mbox{\hyperref[#1]{equation\hspace{2pt}}\ref{#1}}}
\newcommand{\figref}[1]{\mbox{\hyperref[#1]{figure\hspace{2pt}}\ref{#1}}}
\newcommand{\tableref}[1]{\mbox{\hyperref[#1]{table\hspace{2pt}}\ref{#1}}}

% Capitalised Commands
\newcommand{\Appref}[1]{\mbox{\hyperref[#1]{Appendix\hspace{2pt}}\ref{#1}}}
\newcommand{\Eqref}[1]{\mbox{\hyperref[#1]{Equation\hspace{2pt}}\ref{#1}}}
\newcommand{\Figref}[1]{\mbox{\hyperref[#1]{Figure\hspace{2pt}}\ref{#1}}}
\newcommand{\Tableref}[1]{\mbox{\hyperref[#1]{Table\hspace{2pt}}\ref{#1}}}


% use \cite and \textcite to do references




% ---------------------------------------------------------------------------------------------------
% || --------------------------------- HYPERLINK COLOURING --------------------------------------- ||
% ---------------------------------------------------------------------------------------------------

% overridden by customization below
\usepackage{hyperref,xcolor}
\definecolor{darkblue}{rgb}{0,0,0.6}
\hypersetup{
    %pdfpagemode=FullScreen,
    colorlinks=true,
    linkcolor=darkblue,
    filecolor=magenta,      
    urlcolor=blue,
    citecolor=black
}




% ---------------------------------------------------------------------------------------------------
% || ---------------------------------- SECTION FORMATTING --------------------------------------- ||
% ---------------------------------------------------------------------------------------------------

\setlength{\droptitle}{-8em}


%spacing is:  left, before, after
\titlespacing{\section}{0pt}{1em}{-0.5em}
\titlespacing{\subsection}{0pt}{1em}{-0.7em}
\titlespacing{\subsubsection}{0pt}{1em}{-0.8em}
\titlespacing*{\section}{0pt}{5.5ex plus 1ex minus .2ex}{2pt}

\usepackage{bold-extra}
%\titleformat{<command>}[<shape>]{<format>}{<label>}{<sep>}{<before-code>}[<after-code>]
\titleformat{\section}{\normalfont\Large\bfseries}{\thesection}{1em}{}
\titleformat{\subsection}{\normalfont\large\bfseries}{\thesubsection}{1em}{}
\titleformat{\subsubsection}{\normalfont\normalsize\bfseries}{\thesubsubsection}{1em}{}
\titleformat{\paragraph}{\normalfont\normalsize\itshape\bfseries}{{\normalfont\normalsize\bfseries\theparagraph}}{1em}{}

% Add new sublevel
\newcommand{\subsubsubsection}[1]{\paragraph{#1}\mbox{}\\\vspace{-5em}}
% A note - this command only works well if there is an empty line after it


\newcommand{\references}{
    \fancyhead[L]{\headercase{References}}
    \addcontentsline{toc}{section}{References}
    \setlength\bibitemsep{1.5\itemsep}
    \defbibheading{SpecialBibHeadingStyle}{\section*{References}}
    \printbibliography[heading=SpecialBibHeadingStyle]
}



% Use at start of appendix section
\newcommand{\appendices}{
    \appendix
    \titleformat{\section}{\normalfont\Large\bfseries}{\thesection}{1em}{}
    \fancyhead[L]{\headercase{Appendices}}

    \section*{Appendices}
    \addtocounter{section}{1}
    \addcontentsline{toc}{section}{Appendices}

    \titleformat{\subsection}{\normalfont\Large\bfseries}{\thesubsection}{1em}{}
    \titleformat{\subsubsection}{\normalfont\large\bfseries}{\thesubsubsection}{1em}{}
    \titleformat{\paragraph}{\normalfont\normalsize\bfseries}{\paragraph}{1em}{}
}







% ---------------------------------------------------------------------------------------------------
% || ----------------------------------- TITLEPAGE COMMANDS -------------------------------------- ||
% ---------------------------------------------------------------------------------------------------

\newcommand{\topTitle}{
    {
        \centering

        {\LARGE \@title}

        \if \@author \empty
        \else
            {\large \@author}\\
        \fi
    }
    \vspace{3em}
}




\newcommand{\fullPageTitle}{
    \begin{titlepage}

        \vspace{\stretch{1}}

        \newcommand{\HRule}{\rule{\linewidth}{0.5mm}} % Defines a new command for the horizontal lines, change thickness here
        
        \center % Center everything on the page
         

        % HEADINGS -----------------------------------------------------------------------------------
        \vspace*{\fill}
        \if \Uni \empty
        \else
            {\scshape\huge ~\Uni}\\[1em] % Name of your university/college
        \fi
        
        \if \School \empty
        \else
            {\scshape\Large ~\School}\\[4em] % Major heading such as course name
        \fi
        
        % \if \Unit \empty
        % \else
        %     {\scshape\Large ~\Unit}\\[1em]
        % \fi

        \if \Class \empty
        \else
            {\scshape\large ~\Class}\\[3em]% Minor heading such as course title
        \fi


        % TITLE -----------------------------------------------------------------------------------
        \HRule \\[0.4cm]
        { \Huge \bfseries ~\@title}\\[0.1cm] % Title of your document
        \HRule \\[2em]
         

        % YEAR ----------------------------------------------------------------------------------
        \if \Year \empty
        \else
            {\Large \Year}\\[1em]
        \fi


        % AUTHOR ----------------------------------------------------------------------------------
        \if \@author \empty
        \else
            \vspace{2em}
            {\large \@author}
        \fi
        

        \vspace{\stretch{4}} % Fill the rest of the page with whitespace

        % DATE ----------------------------------------------------------------------------------------
        \if \@date \empty
        \else
            {\large ~\@date}\\ % Date, change the \today to a set date if you want to be precise
            \vspace{4em}
        \fi
        
        \end{titlepage}
}




% ---------------------------------------------------------------------------------------------------
% || --------------------------------- MISC. DOCUMENT FORMATTING --------------------------------- ||
% ---------------------------------------------------------------------------------------------------

\let\cleardoublepage=\clearpage % stops random pages being printed after the ToC and titlepage

% Sets the maximum number of levels in the Table of Contents to 3
\setcounter{secnumdepth}{4}


\newcommand{\super}{\textsuperscript}
\newcommand{\sub}{\textsubscript}


% Creates new column type for tables
\newcolumntype{C}[1]{>{\centering\arraybackslash}m{#1}}

% Sets the folder path for images to be a subfolder of where the .tex file is
\graphicspath{{./images/}}

% Headers and footers
\pagestyle{fancy}
\fancyhf{}
\setlength{\headheight}{15.2pt}
\setlength{\headsep}{1.2em}
\renewcommand{\headrulewidth}{0.5pt}
\renewcommand{\footrulewidth}{0.5pt}

% overridden by customisation below
\fancyhead[L]{\headercase{\leftmark}} % Left header - last section on page
\fancyhead[R]{\Unit~Formula Sheet \Year} % Right Header
\fancyfoot[C]{\thepage} % Puts page number bottom centre


\fancypagestyle{noheader}{
  \fancyhf{}% Clear header/footer
  \renewcommand{\headrulewidth}{0pt}% No header rule
  \fancyfoot[RE, RO]{\thepage}
}



% Equation Explanation
\newcommand{\eqexp}[1]{
    \begin{adjustwidth}{0.5cm}{1cm}
        \textsl{#1}
    \end{adjustwidth}
    \vspace{0.5em}
}


\newcommand{\alignColumns}{\switchcolumn*\switchcolumn}






% -------------------------------------------------------------
%   --- New LaTeX users feel free to edit what is below ---   |
% -------------------------------------------------------------

% Making any of these fields blank will remove them from the titlepage
\newcommand{\Uni}{The University of Sydney}
\newcommand{\School}{School of Physics}
\newcommand{\Unit}{Unit Code}
\newcommand{\Year}{Year}
\newcommand{\Class}{}
\newcommand{\Assignment}{}
\newcommand{\SID}{123456789}

\author{Author 1 \\[0.5em] Author 2 \\[0.5em] Author 3 \\}

\date{} % makes it undated, remove this line if you want a date to appear
%\date{\today} % make the date the current date





% ----------------------------------------------------------------------
%   --- Semi-Advanced LaTeX users feel free to edit what is below ---  |
% ----------------------------------------------------------------------

\title{\Unit~Formula Sheet}

% Formatting
\numberwithin{equation}{section} % numbers equation with secNum.EqNum, remove to number globally

\setlength{\parindent}{0em} % indent when there is a double newline within the tex code
\setlength{\parskip}{1em} % vertical space when there is a double newline within the text code


\usepackage{hyperref,xcolor}
\hypersetup{
    colorlinks=true,
    linkcolor=darkblue,
    filecolor=magenta,      
    urlcolor=blue,
    citecolor=black
}


\newcommand{\headercase}[1]{\scshape\nouppercase{#1}}
\fancyhead[L]{\headercase{\leftmark}} % Left header - last section on page
\fancyhead[R]{\Unit~Formula Sheet \Year} % Right Header
\fancyfoot[C]{\small\thepage} % Puts page number bottom centre



% -----------------------------------------------------------------
%   --- Advanced LaTeX users... do what you want ¯\_(ツ)_/¯ ---   |
% -----------------------------------------------------------------